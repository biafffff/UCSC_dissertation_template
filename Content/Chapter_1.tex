
\pagestyle{fancy}

\chapter{\LaTeX \, Use}
\label{ch:latex}

\section{Sectioning}
\label{example-Section}

In order to create a new section use the command \code{\textbackslash section\{\}}. Similarly you can use \code{\textbackslash subsection\{\}} and \code{\textbackslash subsubsection\{\}}. If you want obtain an unnumbered a section write \code{\textbackslash section*\{\}}.

\subsection{Sub Section}
\subsection*{Unnumbered Section}

\section{Listings}

Numbered list. Obtained using the \textit{enumerate} environment.
\begin{verbatim}
\begin{enumerate}
	\item Example 
	\item Example 
	\item Example	
\end{enumerate}
\end{verbatim}

\begin{enumerate}
	\item Example 
	\item Example 
	\item Example	
\end{enumerate}

Itemized list. Obtained using the \textit{itemize} environment.
\begin{itemize}
	\item Example
	\item Example
	\item Example	
\end{itemize}


\section{Math Environment}

The star after the command, avoid printing the equation tag (\code{\textbackslash equation\{\}}). Without the equation will be numbered and can be referenced (see Equation \eqref{example-1}). You can reference also sections (see Section \ref{example-Section}).

The expected value of $X$ is
\begin{align*}
	\mathbb{E}[X] &= \int_{1}^{c} x \; F(x) \; dx \\[0.25cm]
				  &= \int_{1}^{c} x \ \frac{1}{x} \ \frac{1}{log(c)} \; dx \\[0.25cm]
				  &= \frac{1}{log(c)} \int_{1}^{c} dx = \frac{c - 1}{log(c)} \\
	\label{example-1}			  
\end{align*}

\begin{verbatim}
\begin{gather*}
	y =  \omega + \beta_1 x_1 + \beta_2 x_2
\end{gather*}
\end{verbatim}
\begin{gather*}
	y =  \omega + \beta_1 x_1 + \beta_2 x_2
\end{gather*}

\section{Tables}

To create tables use this useful website \href{http://www.tablesgenerator.com}{www.tablesgenerator.com}. It is preferable to use the \textit{Booktabs table style} option as for example in Table \ref{tab:regression}.

\begin{table}[!h]
\centering
\caption{Linear Regression Model 1}
\label{tab:regression}
\begin{tabular}{@{}llrrlrl@{}}
\toprule
\multicolumn{1}{r}{\textbf{}} &  & \textbf{Estimate} & \textbf{Std. Error} & \multicolumn{1}{r}{\textbf{t value}} & \textbf{Pr(\textgreater | t |)} & \textbf{} \\ \midrule
\multicolumn{1}{r}{(Intercept)} &  & 119.7719 & 0.5723 & \multicolumn{1}{r}{209.27} & 0.0000 &  \\
\multicolumn{1}{r}{x} &  & -0.5138 & 0.0905 & \multicolumn{1}{r}{-5.68} & 0.0000 &  \\
 &  & \multicolumn{1}{l}{} & \multicolumn{1}{l}{} &  & \multicolumn{1}{l}{} &  \\
\multicolumn{7}{l}{Residual standard error: 20.13 on 1386 degrees of freedom} \\
\multicolumn{4}{l}{Multiple R-squared: 0.02273} & \multicolumn{3}{l}{Adjusted R-squared: 0.02202} \\
\multicolumn{4}{l}{F-statistic: 32.24 on 1 and 1386 DF} & \multicolumn{3}{l}{p-value: 1.662e-08} \\ \bottomrule
\end{tabular}
\end{table}

\section{Figures}

Please store all your images and graphs in the \textit{Images} folder. \newline
To add figures use the following line of command:

\begin{verbatim}
\begin{figure}
\centering
\includegraphics[width=0.5\textwidth]{Logo-mono.png}
\caption{Example}
\label{fig:logo}
\end{figure}
\end{verbatim}

\begin{figure}[!h]
\centering
\includegraphics[width=0.25\textwidth]{Logo-mono.png}
\caption{Example}
\label{example-figure}
\end{figure}

\section{Bibliography}

You can cite other papers/publications with the command \code{\textbackslash textcite\{\}}. For example writing \code{\textbackslash textcite\{Bianchi2016\}} I obtain this citation: \textcite{Bianchi2016}, that will be printed automatically in alphabetical order in the \textit{Bibliography} at the end of the work.

\section{Troubleshooting}

In case of issues with your \LaTeX$\,$ code please visit this useful website full of questions and answers: \href{https://tex.stackexchange.com}{www.tex.stackexchange.com}.







