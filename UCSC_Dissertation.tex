
% Dissertation Template Università Cattolica del Sacro Cuore

\documentclass[a4paper, 11pt, twoside, openright]{book}

\usepackage[utf8]{inputenc}
\usepackage[T1]{fontenc}
\usepackage[english]{babel}
\usepackage[autostyle, english = american]{csquotes}
%\MakeOuterQuote{"}

\usepackage[hmarginratio=1:1]{geometry}

\usepackage{graphicx}
\graphicspath{{Images/}}

\usepackage{array}
\usepackage{tabularx}
\usepackage{longtable}
\usepackage{subfigure}
\usepackage{amsmath, amsfonts, amssymb, amsthm}
\usepackage{mathtools}          
\usepackage{pdflscape}
%\usepackage{fncychap} % Conny or (better) nothing
%\mghrulefill{2pt}
%\usepackage{caption} 
\usepackage[labelfont=bf]{caption}
\usepackage{float}
\usepackage{adjustbox}
\usepackage[official]{eurosym}
\usepackage{booktabs}
\usepackage{flafter}
\usepackage{indentfirst} % To indent first lines

\usepackage[style=authoryear, isbn=false, url=false, maxcitenames=2]{biblatex} %maxnames=1
\renewbibmacro{in:}{}
\addbibresource{Bibliography.bib}

%\usepackage{tocloft}
%\setcounter{tocdepth}{1} % Only show chapters and sections in ToC

\usepackage{color}
\usepackage{xcolor}
\usepackage{framed}
\usepackage{alltt}

\usepackage{listings}
\usepackage[shortlabels]{enumitem}
%\usepackage{enumerate}

\usepackage[bottom]{footmisc}
\interfootnotelinepenalty=10000

\usepackage{tikz}
\usepackage{environ}

\usepackage{multirow}
\usepackage{xfrac}
\usepackage{nicefrac}
\usepackage{bigstrut}
\usepackage{soul}
\usepackage[all]{nowidow}
\widowpenalty=1000
\usepackage{verbatim}

%\linespread{1.1}
\usepackage[doublespacing]{setspace}

% Title Page Definition
\makeatletter
\def\clap#1{\hbox to 0pt{\hss #1\hss}}%
\def\ligne#1{%
  \hbox to \hsize{%
    \vbox{\centering #1}}}%
\def\haut#1#2#3{%
  \hbox to \hsize{%
    \rlap{\vtop{\raggedright #1}}%
    \hss
    \clap{\vtop{\centering #2}}%
    \hss
    \llap{\vtop{\raggedleft #3}}}}%
\def\bas#1#2#3{%
  \hbox to \hsize{%
    \rlap{\vbox{\raggedright #1}}%
    \hss
    \clap{\vbox{\centering #2}}%
    \hss
    \llap{\vbox{\raggedleft #3}}}}%
\def\maketitle{%
  \thispagestyle{empty}\vbox to \vsize{%
    \haut{}{\@blurb}{}
    \vfill
    \ligne{\Large \@title}
    \vspace{5mm}
    \ligne{\Large \@author}
    \vspace{1mm}\ligne{\texttt{<\@email>}}
    \vspace{1cm}
    \vfill
    \vfill
    \bas{}{\@location, \@date}{}
    }%
  \cleardoublepage
  }
\def\date#1{\def\@date{#1}}
\def\author#1{\def\@author{#1}}
\def\title#1{\def\@title{#1}}
\def\location#1{\def\@location{#1}}
\def\blurb#1{\def\@blurb{#1}}
\def\email#1{\def\@email{#1}}

% Abstract definition
\newenvironment{abstract}%
{\cleardoublepage\null \vfill\begin{center}
\bfseries \abstractname \end{center}}
{\vfill\null}

% Fancy Header
\usepackage{fancyhdr}
\pagestyle{fancy}
\renewcommand{\chaptermark}[1]{\markboth{\MakeUppercase{ %\scshape{
\chaptername}\ \thechapter:%
 \ #1}{}}
%\renewcommand{\sectionmark}[1]{%
% \markboth{\sectionname
% \ \thesection.\ #1}{}}
%\renewcommand{\sectionmark}[1]{\markboth{\sectionname \ #1}{}}

% OR \footnotesize !!!
\fancyhf{}
\fancyhead[LE]{\thepage \hspace{0.5cm} \leftmark} 
\fancyhead[RO]{\nouppercase \rightmark \hspace{0.5cm} \thepage} 
%\renewcommand{\headrulewidth}{0.3pt}
\renewcommand{\headrulewidth}{0pt}

\fancypagestyle{mypagestyle}{
\fancyhf{}
\cfoot{\thepage}
\renewcommand{\footrulewidth}{0pt}}

% Avoid header/footer on blank pages
\makeatletter
\def\cleardoublepage{\clearpage\if@twoside \ifodd\c@page \else\hbox{}\thispagestyle{empty}\newpage
\if@twocolumn\hbox{}\newpage\fi\fi\fi} \makeatother

\usepackage{hyperref}
%\usepackage[backref=page]{hyperref}

\theoremstyle{plain}
\newtheorem{thm}{Theorem}[chapter] % reset theorem numbering for each chapter

\theoremstyle{definition}
\newtheorem{defn}[thm]{Definition} % definition numbers are dependent on theorem numbers
\newtheorem{exmp}[thm]{Example}
\newtheorem{rmk}[thm]{Remark}
\newtheorem{prop}[thm]{Proposition}
\newtheorem{coroll}[thm]{Corollary}

\usepackage{txfonts}
\usepackage{bbm}
\usepackage{pxfonts}

% Chapter style 

%\makeatletter
%\def\thickhrulefill{\leavevmode \leaders \hrule height 1ex \hfill \kern \z@}
%\def\@makechapterhead#1{%
%  \vspace*{10\p@}%
%  {\parindent \z@ 
%    {\raggedleft \reset@font
%      \scshape \@chapapp{} \thechapter\par\nobreak}%
%    \par\nobreak
%    \vspace*{30\p@}
%    \interlinepenalty\@M
%    {\raggedright \Huge \bfseries #1}%
%    \par\nobreak
%    \hrulefill
%    \par\nobreak
%    \vskip 100\p@
%  }}
%\def\@makeschapterhead#1{%
%  \vspace*{10\p@}%
%  {\parindent \z@ 
%    {\raggedleft \reset@font
%      \scshape \vphantom{\@chapapp{} \thechapter}\par\nobreak}%
%    \par\nobreak
%    \vspace*{30\p@}
%    \interlinepenalty\@M
%    {\raggedright \Huge \bfseries #1}%
%    \par\nobreak
%    \hrulefill
%    \par\nobreak
%    \vskip 100\p@
%  }}
  
\makeatletter
\def\thickhrulefill{\leavevmode \leaders \hrule height 1ex \hfill \kern \z@}
\def\@makechapterhead#1{%
  \reset@font
  \vspace*{10\p@}%
  {\parindent \z@ 
    \begin{flushleft}
    %\bfseries
      \reset@font \scshape \LARGE \scshape \@chapapp{} \thechapter \par
    \end{flushleft}
    \hrule
    \begin{flushleft}
      \reset@font \bfseries \Huge \strut #1\strut \par
    \end{flushleft}
    \vskip 100\p@
  }}
\def\@makeschapterhead#1{%
  \reset@font
  \vspace*{10\p@}%
  {\parindent \z@ 
    \begin{flushleft}
    %\bfseries
      \reset@font \scshape \LARGE \vphantom{\thechapter} \par
    \end{flushleft}
    \hrule
    \begin{flushleft}
      \reset@font \bfseries \Huge \strut #1\strut \par
    \end{flushleft}
    \vskip 100\p@
  }}

%\makeatletter
%\def\thickhrulefill{\leavevmode \leaders \hrule height 1ex \hfill \kern \z@}
%\def\@makechapterhead#1{%
%  \vspace*{10\p@}%
%  {\parindent \z@ 
%        \raggedleft
%        \reset@font\huge\bfseries
%        \begin{tabular}{c|p{15cm}}
%          {\qquad\thechapter{}\  }
%          &\quad
%          \Huge #1
%        \end{tabular}
%        \par\nobreak
%    \vskip 100\p@
%  }}
%\def\@makeschapterhead#1{%
%  \vspace*{10\p@}%
%  {\parindent \z@ 
%        \raggedleft
%        \reset@font\huge\bfseries
%        \begin{tabular}{cp{15cm}}
%          {\qquad\hphantom{\thechapter{}}\  }
%          &\quad
%          \Huge #1
%        \end{tabular}
%        \par\nobreak
%    \vskip 100\p@
%  }}

% To add equation tag to environment*
\newcommand\numberthis{\addtocounter{equation}{1}\tag{\theequation}}

% To set the first letter of each chapter bigger - Journal style
\usepackage{lettrine}

% To create a overbar line bigger than \bar and shorter than \overline
\newcommand{\overbar}[1]{\mkern 1.5mu\overline{\mkern-1.5mu#1\mkern-1.5mu}\mkern 1.5mu}

% To scale equation too long to fir the margins
\newcommand\scalemath[2]{\scalebox{#1}{\mbox{\ensuremath{\displaystyle #2}}}}

\usepackage{lipsum}

\newcommand{\code}[1]{\texttt{#1}}

%%%%%%%%%%%%%%%%%%%%%%%%%%%%%%%%%%%%%%%%%%%%%%%%%%%%%%%%%%%%%%%%%

\begin{document}

%\begin{frontmatter}
\frontmatter
%\pagestyle{fancy}
%\fncyfront
	
\pagestyle{empty}

\vspace*{-1.5cm}

\begin{center}
\large
UNIVERSIT\`{A} CATTOLICA DEL SACRO CUORE \\
\normalsize
Facoltà di Scienze Bancarie, Finanziarie e Assicurative \\
Corso di laurea in Banking and Finance \\

\begin{figure}[H]
  \centering
	\includegraphics[width=2.25cm]{Logo-mono.png}
  \end{figure}
\vspace*{-0.6cm}
\noindent\rule{4cm}{0.2pt}
\vspace*{2cm}
  
%\Large{TITLE}
\huge{\textbf{\textsc{{Title}}}

\vspace*{.75truecm}
\end{center}

\vfill
\vfill

\large

\begin{flushright}
\emph{Graduand}: Name Surname \\
\emph{Student ID}: 12345678910 \\
\end{flushright}

\noindent \emph{Supervisor}: Professor Name Surname \\
% \emph{Co-Supervisor}: Professor Name Surname 
% Optional

%\vspace*{1.0cm}

%\begin{flushright}
%\emph{Graduand}: Name Surname \\
%\end{flushright}

\vfill
\vfill

\begin{center}
  Academic Year 20**$-$20**
\end{center}
 			% not on TOC
	\include{Content/Copyright_Notice}		% not on TOC
	
\pagestyle{empty}

\newpage\null\thispagestyle{empty}\newpage

\null\vspace{\stretch{1}}
\begin{flushright}
	\textit{Dedication}
\end{flushright}
\vspace{\stretch{2}}\null			% not on TOC
	%\addcontentsline{toc}{chapter}{Abstract}
	
\thispagestyle{mypagestyle}

\begin{abstract}
\begin{center}

%\textsc{The role of conditioning information: \\ an investment perspective} \\[7pt]

Title \\[7pt]

\emph{Author} \\[5pt]

Professor Name Surname \\
% $\quad$ Professor Name Surname \\ Uncomment this line if you want to show the name of the co-supervisor

\end{center}

No more than 350 words. It is normally a single paragraph, consists of four parts: the statement of the problem; the procedure and methods used to investigate the problem; the results of the investigation; and the conclusions. \\

\noindent \textit{Keywords}: \\

\end{abstract}				% not on TOC
	
\newpage\null\thispagestyle{empty}\newpage

\thispagestyle{mypagestyle}

\null\vspace{\stretch{1}}

I acknowledge a special debt to ...

\vspace{\stretch{2}}\null		% not on TOC
	\tableofcontents
	%\listoffigures 
	%\listoftables
	\include{Content/Preface}	
%\end{frontmatter}

% The float package, with the \newfloat and \listof commands, can be used to create lists of custom floating objects (e.g. programs, algorithms, etc.). The tocloft package can be used to modify their layout.

%\begin{mainmatter}
\mainmatter
%\fncymain
	
\pagestyle{fancy}

\chapter*{Introduction}
\addcontentsline{toc}{chapter}{Introduction}
\label{Introduction}

\textbf{Please cancel the faculties list once modified the title page and add your own text}. \\

\begin{itemize}
	\item Economia
	\item Economia e Giurisprudenza
	\item Giurisprudenza
	\item Lettere e Filosofia
	\item Medicina e Chirurgia
	\item Psicologia
	\item Scienze Agrarie, Alimentari e Ambientali
	\item Scienze Bancarie, Finanziarie e Assicurative
	\item Scienze della Formazione
	\item Scienze Linguistiche e Letterature Straniere
	\item Scienze Matematiche, Fisiche e Naturali
	\item Scienze Politiche e Sociali
\end{itemize}


% Cancel the faculties list and add your text
	
\pagestyle{fancy}

\chapter{\LaTeX \, Use}
\label{ch:latex}

\section{Sectioning}
\label{example-Section}

In order to create a new section use the command \code{\textbackslash section\{\}}. Similarly you can use \code{\textbackslash subsection\{\}} and \code{\textbackslash subsubsection\{\}}. If you want obtain an unnumbered a section write \code{\textbackslash section*\{\}}.

\subsection{Sub Section}
\subsection*{Unnumbered Section}

\section{Listings}

Numbered list. Obtained using the \textit{enumerate} environment.
\begin{verbatim}
\begin{enumerate}
	\item Example 
	\item Example 
	\item Example	
\end{enumerate}
\end{verbatim}

\begin{enumerate}
	\item Example 
	\item Example 
	\item Example	
\end{enumerate}

Itemized list. Obtained using the \textit{itemize} environment.
\begin{itemize}
	\item Example
	\item Example
	\item Example	
\end{itemize}


\section{Math Environment}

The star after the command, avoid printing the equation tag (\code{\textbackslash equation\{\}}). Without the equation will be numbered and can be referenced (see Equation \eqref{example-1}). You can reference also sections (see Section \ref{example-Section}).

The expected value of $X$ is
\begin{align*}
	\mathbb{E}[X] &= \int_{1}^{c} x \; F(x) \; dx \\[0.25cm]
				  &= \int_{1}^{c} x \ \frac{1}{x} \ \frac{1}{log(c)} \; dx \\[0.25cm]
				  &= \frac{1}{log(c)} \int_{1}^{c} dx = \frac{c - 1}{log(c)} \\
	\label{example-1}			  
\end{align*}

\begin{verbatim}
\begin{gather*}
	y =  \omega + \beta_1 x_1 + \beta_2 x_2
\end{gather*}
\end{verbatim}
\begin{gather*}
	y =  \omega + \beta_1 x_1 + \beta_2 x_2
\end{gather*}

\section{Tables}

To create tables use this useful website \href{http://www.tablesgenerator.com}{www.tablesgenerator.com}. It is preferable to use the \textit{Booktabs table style} option as for example in Table \ref{tab:regression}.

\begin{table}[!h]
\centering
\caption{Linear Regression Model 1}
\label{tab:regression}
\begin{tabular}{@{}llrrlrl@{}}
\toprule
\multicolumn{1}{r}{\textbf{}} &  & \textbf{Estimate} & \textbf{Std. Error} & \multicolumn{1}{r}{\textbf{t value}} & \textbf{Pr(\textgreater | t |)} & \textbf{} \\ \midrule
\multicolumn{1}{r}{(Intercept)} &  & 119.7719 & 0.5723 & \multicolumn{1}{r}{209.27} & 0.0000 &  \\
\multicolumn{1}{r}{x} &  & -0.5138 & 0.0905 & \multicolumn{1}{r}{-5.68} & 0.0000 &  \\
 &  & \multicolumn{1}{l}{} & \multicolumn{1}{l}{} &  & \multicolumn{1}{l}{} &  \\
\multicolumn{7}{l}{Residual standard error: 20.13 on 1386 degrees of freedom} \\
\multicolumn{4}{l}{Multiple R-squared: 0.02273} & \multicolumn{3}{l}{Adjusted R-squared: 0.02202} \\
\multicolumn{4}{l}{F-statistic: 32.24 on 1 and 1386 DF} & \multicolumn{3}{l}{p-value: 1.662e-08} \\ \bottomrule
\end{tabular}
\end{table}

\section{Figures}

Please store all your images and graphs in the \textit{Images} folder. \newline
To add figures use the following line of command:

\begin{verbatim}
\begin{figure}
\centering
\includegraphics[width=0.5\textwidth]{Logo-mono.png}
\caption{Example}
\label{fig:logo}
\end{figure}
\end{verbatim}

\begin{figure}[!h]
\centering
\includegraphics[width=0.25\textwidth]{Logo-mono.png}
\caption{Example}
\label{example-figure}
\end{figure}

\section{Bibliography}

You can cite other papers/publications with the command \code{\textbackslash textcite\{\}}. For example writing \code{\textbackslash textcite\{Bianchi2016\}} I obtain this citation: \textcite{Bianchi2016}, that will be printed automatically in alphabetical order in the \textit{Bibliography} at the end of the work.

\section{Troubleshooting}

In case of issues with your \LaTeX$\,$ code please visit this useful website full of questions and answers: \href{https://tex.stackexchange.com}{www.tex.stackexchange.com}.








	\pagestyle{fancy}

\chapter{Chapter 2}
\label{ch:2}
	\pagestyle{fancy}

\chapter{Chapter 3}
\label{ch:3}
	\pagestyle{fancy}

\chapter{Chapter 4}
\label{ch:4}
	\appendix
	
\thispagestyle{empty}

\chapter{Appendix}
\label{appendixA}

This appendix shows part of code written to $\hdots$

	\include{Content/Appendix_B}	
%\end{mainmatter}

%\begin{backmatter}
\backmatter
	\addcontentsline{toc}{chapter}{Bibliography}
	\printbibliography	
%\end{backmatter}

\end{document}